\documentclass{article}[14pt]
\usepackage[export]{adjustbox}[2011/08/13]
\usepackage{graphicx}
\usepackage{hyperref}
\begin{document}
\title{Bachelor Project Plan \\ Terrain traversability prediction by a robot}
\author{Stefano Peverelli \\ Adv. Luca Maria Gambardella \\ TAs Alessandro Giusti, Jerome Guzzi}
\date{\today}
\maketitle
\section*{Introduction}
The project aims to use Machine Learning in order for a robot to predict whether it can traverse a given terrain or not. 
The dataset passed to the Neural Network will be the result of several simulations in VREP, an open-source virtual robot experimentation platform.
\section*{Motivation and Goal}
Unmanned vehicle navigation and terrain traversability is a core issue in robotics and in its consequent applications such as 
land exploration, military scouting or reconnaissance  and more general situations, thus the need for an intelligent system,
that can guarantee a \textbf{wheeled} robot to estimate with accuracy if a given terrain patch is traversable.
\section*{Project Tasks}
There are two main phases in the project: 
\begin{enumerate}
\item
    The goal of this phase consists in writing a controller for the simulated robot and building a dataset
    of terrain labels that the robot has either traversed or not.
\item
    The goal of this phase is to build a classifier (such as a Neural Network) to predict the traversability of a given terrain patch.
\end{enumerate}
\subsection*{Writing the controller and building a Dataset}
The very first task is write a controller for the simulated robot (in such a way that the robot follow a predetermined path), and analyze what happened
during the simulation (the robot got stuck or capsized).
In order to obtain an acceptable final result a key component such as the dataset must be construct. For this application
the dataset will consist of a collection of thousands hundreds of patch label obtained by several simulations with V-REP. 
\subsubsection*{Building the 3d terrain}
In order to simulate an off road environment and guarantee a large dataset of possible terrain configurations that the robot will face,
a large dataset of \textbf{heightmaps} will be built. The process of generating a 3d terrain in a procedural way is based on the Perlin Noise algorithm,
\url{https://en.wikipedia.org/wiki/Perlin_noise}\\\\
\includegraphics*[scale=0.05]{heightmap.png}
\includegraphics*[scale=0.185]{3dterrain.png}
\subsection*{Build a Classifier}
After collecting the results from the simulations, the final task is to analyze the dataset with supervisors and investigate options to apply machine learning
to automatically estimate whether a given area is traversable or not.\\\\
\begin{center}
    \includegraphics*[scale=0.075]{neuralnetwork.png}
\end{center}
\section*{Planning}
\includegraphics*[width=1.4\textwidth,center]{timetable}\\\\
Above is the work planning, in red Milestones are highlighted. The thesis will be written between the $10^{th}$ and the $14^{th}$ week.
\end{document}

