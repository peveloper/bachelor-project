\documentclass{article}[12pt]
\begin{document}
\title{Bachelor Project - Abstract}
\author{Stefano Peverelli}
\date{\today}
\maketitle
Unmanned vehicle navigation is a core topic in robotics, thus the need for an intelligent system to 
address the issue of path planning for ground robots on offroad terrain. \\
The planned approach expects two main phases: the first one is to simulate an offroad wheeled
ground robot that traverses a 3d given terrain model in a virtual environment, analyze in which areas it
got stuck or capsized, and collect data from the simulations to obtain an extensive
dataset of possible outcomes, which comprehends various information such as the elapsed time,
whether the robot has traversed the patch or not, the goal's position and the robot's initial and
final position. \\
The second phase instead aims to analyze the resulting dataset and use it to train a neural network,
which final goal is, as mentioned above, to foretell the \textit{traversabiity} of a terrain (by its 3d structure),
planning a path the robot can traverse.
By using an approach based on machine learning, there's no need to manually define heuristic
rules to predict the traversability of terrain patches; moreover, the approach is applicable
to any type of ground robot (wheeled/legged) as long as it can be reliably simulated.
\end{document}



